% Copyright (C) 2012 Raniere Silva
% 
% This work is licensed under the Creative Commons
% Attribution-ShareAlike 3.0 Unported License. To view a copy of this
% license, visit <http://creativecommons.org/licenses/by-sa/3.0/>.
% 
% This work is distributed in the hope that it will be useful, but
% WITHOUT ANY WARRANTY; without even the implied warranty of
% MERCHANTABILITY or FITNESS FOR A PARTICULAR PURPOSE.

\documentclass[11pt]{beamer}
% Utilizar apenas para a classe beamer
\usetheme{CambridgeUS}
\let\Tiny=\tiny % Redefine at least \Tiny for avoid warning

% Tipo de arquivo.
\usepackage[utf8]{inputenc}
% \usepackage[latin1]{inputenc}
\usepackage[T1]{fontenc}

% Configura\c{c}\~{o}es regionais
% \usepackage[top=3cm,left=2cm,right=2cm,bottom=3cm]{geometry}
\usepackage[brazil]{babel}
\usepackage{indentfirst}
\uselanguage{brazil}
\languagepath{brazil}
\deftranslation[to=brazil]{Example}{Exemplo}

% Textos
\newcommand{\flang}[1]{\textit{#1}}

% Links
\usepackage{url}
\usepackage{hyperref}
\hypersetup{
%colorlinks = true,
}
\usepackage{breakurl}

% Pacotes matem\'{a}ticos
\usepackage{amsmath}
\usepackage{amsfonts}
\usepackage{amssymb}
\usepackage{amsthm}
\allowdisplaybreaks[4]
\newtheorem{defi}{Definição}
\newtheorem{prop}{Proposição}

% Pacotes para tabelas
\usepackage{multicol}
\usepackage{multirow}
\usepackage{array}

% Pacotes gr\'{a}ficos
\usepackage{graphicx}
\usepackage{subfigure}
\usepackage{wrapfig}
\usepackage{tikz}
\usetikzlibrary{positioning}
\usetikzlibrary{fit}

% Pacotes para algoritmos
\usepackage{algorithmic}
\algsetup{linenosize=\small}
\renewcommand{\algorithmicrequire}{\textbf{Entrada:}}
\renewcommand{\algorithmicensure}{\textbf{Saída:}}
\renewcommand{\algorithmicend}{\textbf{fim}}
\renewcommand{\algorithmicif}{\textbf{se}}
\renewcommand{\algorithmicthen}{\textbf{ent\~{a}o}}
\renewcommand{\algorithmicelse}{\textbf{caso contr\'{a}rio}}
\renewcommand{\algorithmicendif}{\algorithmicend}
\renewcommand{\algorithmicfor}{\textbf{para}}
\renewcommand{\algorithmicforall}{\textbf{para todo}}
\renewcommand{\algorithmicdo}{\textbf{fa\c{c}a}}
\renewcommand{\algorithmicendfor}{\algorithmicend}
\renewcommand{\algorithmicwhile}{\textbf{enquanto}}
\renewcommand{\algorithmicendwhile}{\algorithmicend}
\renewcommand{\algorithmicrepeat}{\textbf{repita}}
\renewcommand{\algorithmicuntil}{\textbf{at\'{e}}}
\renewcommand{\algorithmicreturn}{\textbf{retorne}}
\renewcommand{\algorithmiccomment}[1]{\hspace{2em}/* #1 */}

\usepackage{algorithm}
\floatname{algorithm}{Algoritmo}

% Pacotes para c\'{o}digos
\usepackage{textcomp}
\usepackage{listings}
\renewcommand{\lstlistingname}{C\'{o}digo}
\lstset{
% language=Octave,
basicstyle=\ttfamily\scriptsize,
columns=flexible,
% numbers=left,
% numberstyle=\footnotesize,
% stepnumber=5,
% numbersep=5pt,
% backgroundcolor=\color{white},
% showspaces=false,
% showstringspaces=false,
% showtabs=false,
% frame=single,
tabsize=4,
captionpos=t,
breaklines=true,
breakatwhitespace=false,
% caption={\texttt{\lstname}},
% escapeinside={\%*}{*)},
% morekeywords={#},
upquote=true,
}
\newcommand{\lcode}[1]{\lstinline!#1!}
% Configura\{c}c\~{o}es para o python
\lstdefinestyle{python}{
language=python,
% escapeinside={\%}{\^{M},
}
\lstnewenvironment{cpython}{\lstset{style=python,}}{}
\newcommand{\fpython}[1]{\lstinputlisting[style=python,]{#1}}

% Pacotes para incluir arquivos
\usepackage{attachfile}

% Index
\usepackage{makeidx}
\makeindex

\begin{document}
\title{Uso Eficiente do Git}
\author[Raniere Silva]{Raniere Gaia Costa da
Silva\footnote{r.gaia.cs@gmail.com}}

\begin{frame}
    \maketitle
\end{frame}

\begin{frame}
    Esta apresentação é baseada na apresentação ``Intro to Git'' de Scott
    Chacon que encontra-se disponível em
    \url{https://github.com/schacon/git-presentations} e no ``Git Ready''
    (\url{http://gitready.com/}) de Nick Quaranto.

    \begin{block}{}
        Os arquivos desta apresentação encontram-se disponíveis \\
        em \url{https://github.com/r-gaia-cs/presentations}. \\
        \vspace{-33pt}
        \begin{flushright}
            \includegraphics[height=2cm]{../figures/forkme_right_red.png}
        \end{flushright}
    \end{block}

    \begin{block}{Licença}
        Salvo indicado o contrário, esta apresentação está licenciada sob a licença
        Creative Commons Atribuição 3.0 Não Adaptada. Para ver uma cópia desta
        licença, visite http://creativecommons.org/licenses/by/3.0/.
        \begin{center}
            \includegraphics{../figures/cc-by-sa.png}
        \end{center}
    \end{block}
\end{frame}

\begin{frame}
    \tableofcontents
\end{frame}

\section{Refazendo o último commit}

\section{Juntando branchs}
\begin{frame}{Merge}
    \begin{center}
        \begin{tikzpicture}
            \useasboundingbox (-1,0) rectangle (6,4);
            \node [draw, circle] (C0) at (0,0) {C0};
            \node [draw, circle] (C1) at (1,0) {C1};
            \draw  (C0) -- (C1);
            \node [draw, circle] (C2) at (2,1) {C2};
            \draw  (C1) -- (C2);
            \node [draw, circle] (C3) at (3,1) {C3};
            \draw  (C2) -- (C3);
            \node [draw, circle] (C4) at (4,0) {C4};
            \draw  (C1) -- (C4);
            \only<1>{
                \draw[->] (5,2) node[fill=yellow] {master} -| (C4);
                \draw[->] (5,3) node[fill=yellow] {head} -| (C4);
                \draw[->] (5,4) node[fill=gray] {sand} -| (C3);
            }
            \node<.(2)->[draw, circle] (C5) at (5,0) {C5};
            \draw<.(2)-> (C4) -- (C5);
            \draw<.(2)-> (C3) to[out=0] (C5);
            \only<2>{
                \draw[->] (6,2) node[fill=yellow] {master} -| (C5);
                \draw[->] (6,3) node[fill=yellow] {head} -| (C5);
                \draw[->] (6,4) node[fill=gray] {sand} -| (C5);
            }
        \end{tikzpicture}
    \end{center}
    \only<2>{\attachfile[icon=Paperclip, mimetype=text/plain]{examples/merge.sh} git merge sand}
\end{frame}

\begin{frame}{Rebase}
    \begin{center}
        \begin{tikzpicture}
            \useasboundingbox (-1,0) rectangle (6,4);
            \node [draw, circle] (C0) at (0,0) {C0};
            \node [draw, circle] (C1) at (1,0) {C1};
            \draw  (C0) -- (C1);
            \only<1>{
                \node [draw, circle] (C2) at (2,1) {C2};
                \draw  (C1) -- (C2);
                \node [draw, circle] (C3) at (3,1) {C3};
                \draw  (C2) -- (C3);
                \node [draw, circle] (C4) at (4,0) {C4};
                \draw  (C1) -- (C4);
                \draw[->] (5,2) node[fill=yellow] {master} -| (C4);
                \draw[->] (5,3) node[fill=yellow] {head} -| (C4);
                \draw[->] (5,4) node[fill=gray] {sand} -| (C3);
            }
            \only<2>{
                \node [draw, circle] (C2M) at (2,0) {C2};
                \draw  (C1) -- (C2M);
                \node [draw, circle] (C3M) at (3,0) {C3};
                \draw  (C2M) -- (C3M);
                \node [draw, circle] (C4M) at (4,0) {C4};
                \draw  (C3M) -- (C4M);
            }
            \only<2>{
                \draw[->] (6,2) node[fill=yellow] {master} -| (C4M);
                \draw[->] (6,3) node[fill=yellow] {head} -| (C4M);
                \draw[->] (6,4) node[fill=gray] {sand} -| (C3M);
            }
        \end{tikzpicture}
    \end{center}
    \only<2>{\attachfile[icon=Paperclip, mimetype=text/plain]{examples/rebase.sh} git rebase sand}
\end{frame}

% \section{Reorganizando commits}

\section{Agrupando commits}
\begin{frame}{Squash}
    \begin{center}
        \begin{tikzpicture}
            \useasboundingbox (-1,0) rectangle (6,4);
            \node [draw, circle] (C0) at (0,0) {C0};
            \node [draw, circle] (C1) at (1,0) {C1};
            \draw  (C0) -- (C1);
            \only<1>{
                \node [draw, circle] (C2) at (2,1) {C2};
                \draw  (C1) -- (C2);
                \node [draw, circle] (C3) at (3,1) {C3};
                \draw  (C2) -- (C3);
                \node [draw, circle] (C4) at (4,0) {C4};
                \draw  (C1) -- (C4);
                \draw[->] (5,2) node[fill=yellow] {master} -| (C4);
                \draw[->] (5,3) node[fill=yellow] {head} -| (C4);
                \draw[->] (5,4) node[fill=gray] {sand} -| (C3);
            }
            \only<2>{
                \node [draw, circle] (C2M) at (2,0) {C2};
                \draw  (C1) -- (C2M);
                \node [draw, circle] (C3M) at (3,0) {C3};
                \draw  (C2M) -- (C3M);
                \node [draw, circle] (C4M) at (4,0) {C4};
                \draw  (C3M) -- (C4M);
            }
            \only<2>{
                \draw[->] (6,2) node[fill=yellow] {master} -| (C4M);
                \draw[->] (6,3) node[fill=yellow] {head} -| (C4M);
                \draw[->] (6,4) node[fill=gray] {sand} -| (C3M);
            }
        \end{tikzpicture}
    \end{center}
    \only<2>{\attachfile[icon=Paperclip,
    mimetype=text/plain]{examples/squash.sh} git rebase -i}
\end{frame}
\section*{Obrigado}
\begin{frame}
    \begin{center}
        Obrigado!
    \end{center}
    \begin{center}
        \url{r.gaia.cs@gmail.com}
    \end{center}
\end{frame}

\end{document}
