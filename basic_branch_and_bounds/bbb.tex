% Copyright (C) 2012 Raniere Silva
%
% This work is licensed under the Creative Commons
% Attribution-ShareAlike 3.0 Unported License. To view a copy of this
% license, visit <http://creativecommons.org/licenses/by-sa/3.0/>.
%
% This work is distributed in the hope that it will be useful, but
% WITHOUT ANY WARRANTY; without even the implied warranty of
% MERCHANTABILITY or FITNESS FOR A PARTICULAR PURPOSE.

\documentclass[11pt]{beamer}

\usetheme{CambridgeUS}
\let\Tiny=\tiny % Redefine at least \Tiny for avoid warning

\usepackage[utf8]{inputenc}
\usepackage[T1]{fontenc}

\usepackage[brazil]{babel}
\usepackage{indentfirst}
\uselanguage{brazil}
\languagepath{brazil}
\deftranslation[to=brazil]{Example}{Exemplo}

\usepackage{url}
\usepackage{hyperref}
\usepackage{breakurl}

\usepackage{xmpincl}

\usepackage{amsmath}
\usepackage{amsfonts}
\usepackage{amssymb}
\usepackage{amsthm}
\usepackage{breqn}

\usepackage{multicol}
\usepackage{multirow}
\usepackage{array}

\usepackage{graphicx}
\usepackage{subfigure}
\usepackage{wrapfig}
\usepackage{tikz}

\usepackage{algorithmic}
\algsetup{linenosize=\small}

\usepackage{algorithm}
\floatname{algorithm}{Algoritmo}

\usepackage{textcomp}
\usepackage{listings}
\lstset{
% language=Octave,
basicstyle=\ttfamily\scriptsize,
columns=flexible,
% numbers=left,
% numberstyle=\footnotesize,
% stepnumber=5,
% numbersep=5pt,
% backgroundcolor=\color{white},
% showspaces=false,
% showstringspaces=false,
% showtabs=false,
% frame=single,
tabsize=4,
captionpos=t,
breaklines=true,
breakatwhitespace=false,
% caption={\texttt{\lstname}},
% escapeinside={\%*}{*)},
% morekeywords={#},
upquote=true,
}

% Index
\usepackage{makeidx}
\makeindex


\newcommand{\flang}[1]{\textit{#1}}

\hypersetup{
pdftitle={Branch \& Bounds - Básico},
pdfauthor={Raniere Silva},
pdfsubject={Programação Combinatória, Programação Inteira, Branch \& Bounds},
pdfcreator={Raniere Silva},
pdfkeywords={Programação Combinatória, Programação Inteira, Branch \& Bounds},
}
\includexmp{bbb}

\newtheorem{defi}{Definição}
\newtheorem{prop}{Proposição}

\renewcommand{\algorithmicrequire}{\textbf{Entrada:}}
\renewcommand{\algorithmicensure}{\textbf{Saída:}}
\renewcommand{\algorithmicend}{\textbf{fim}}
\renewcommand{\algorithmicif}{\textbf{se}}
\renewcommand{\algorithmicthen}{\textbf{ent\~{a}o}}
\renewcommand{\algorithmicelse}{\textbf{caso contr\'{a}rio}}
\renewcommand{\algorithmicendif}{\algorithmicend}
\renewcommand{\algorithmicfor}{\textbf{para}}
\renewcommand{\algorithmicforall}{\textbf{para todo}}
\renewcommand{\algorithmicdo}{\textbf{fa\c{c}a}}
\renewcommand{\algorithmicendfor}{\algorithmicend}
\renewcommand{\algorithmicwhile}{\textbf{enquanto}}
\renewcommand{\algorithmicendwhile}{\algorithmicend}
\renewcommand{\algorithmicrepeat}{\textbf{repita}}
\renewcommand{\algorithmicuntil}{\textbf{at\'{e}}}
\renewcommand{\algorithmicreturn}{\textbf{retorne}}
\renewcommand{\algorithmiccomment}[1]{\hspace{2em}/* #1 */}

\renewcommand{\lstlistingname}{C\'{o}digo}

\newcommand{\nodebbbtwo}[3]{\begin{tabular}{|c|c|c|} \hline
  $x_0$ & $x_1$ & $f(x)$ \\ \hline
  $#1$ & $#2$ & $#3$ \\ \hline
\end{tabular}}
\newcommand{\nodebounds}[3]{\begin{tabular}{|c|c|c|} \hline
  \multirow{2}{*}{$#1$} & $\overline{z}$ & $#2$ \\ \cline{2-3}
  & $\underline{z}$ &  $#3$ \\ \hline
 \end{tabular}}

\begin{document}
\title[Branch \& Bounds - Básico]{Branch \& Bounds - Básico}
\author[Raniere Silva]{Raniere Gaia Costa da
Silva\footnote{r.gaia.cs@gmail.com}}
\date{22/03/2013}

\begin{frame}
  \maketitle
\end{frame}

\begin{frame}
  \begin{block}{}
    Os arquivos desta apresentação encontram-se disponíveis \\
    em \url{https://github.com/r-gaia-cs/presentations}. \\
    \vspace{-33pt}
    \begin{flushright}
      \includegraphics[height=2cm]{../figures/forkme_right_red.png}
    \end{flushright}
  \end{block}

  \begin{block}{Licença}
    Salvo indicado o contrário, esta apresentação está licenciada sob a
    Licença Creative Commons Atribuição-CompartilhaIgual 3.0 Não Adaptada .
    Para ver uma cópia desta licença, visite
    \url{http://creativecommons.org/licenses/by/3.0/deep.pt_BR}.
    \begin{center}
      \includegraphics{../figures/cc-by-sa.png}
    \end{center}
  \end{block}
\end{frame}

\begin{frame}
    \tableofcontents
\end{frame}

\section{Motivação}
\begin{frame}
  \frametitle{Problemas Reais}
  \begin{itemize}
    \item \flang{Scheduling}
    \item \flang{Planning}
  \end{itemize}
\end{frame}

\begin{frame}
  \frametitle{Exemplos}
  \begin{itemize}
    \item Problema de designação Trabalho-Homem/Máquina
    \item Problema de Cobertura
    \item Problema da Mochila
    \item Problema do Caixeiro Viajante
  \end{itemize}
\end{frame}

\section{Classes de Problemas}
\begin{frame}{PL, PNL e PC}
  \begin{center}
    \begin{tikzpicture}[fill opacity=0.5]
      \useasboundingbox (-4,3) rectangle (6,-3);
      \filldraw<.(1)->[fill=blue] (120:1) circle (1.5) node[opaque, above] {PL};
      \node<.(1)->[opaque] at (160:4) {$\max c^T x$};
      \filldraw<.(2)->[fill=green] (240:1) circle (1.5) node[opaque, below] {PNL};
      \node<.(2)->[opaque] at (200:4) {$\max x^T x$};
      \filldraw<.(3)->[fill=red] (0:1) circle (1.5) node[opaque, right] {PC};
      \node<.(3)->[opaque] at (-20:5) {$\max \left\{ \sum_{i \in I} c_i \mid I \subset \left\{
      1, \ldots, n \right\} \right\}$};
    \end{tikzpicture}
  \end{center}
\end{frame}

\begin{frame}{PC igual a PI?}
  \begin{dmath}[number=PC]
    \max \left\{ \sum_{i \in I} c_i \mid I \subset \left\{ 1, \ldots, n \right\}
    \right\}
  \end{dmath}
  \pause
  \begin{align}
    \text{max } & c^T x \tag{PI} \\
    \text{s.a } & x \in \left\{ 0, 1 \right\}^n \nonumber
  \end{align}
\end{frame}

\section{Modelagem}
\begin{frame}{Exemplo}
  \begin{align}
    \text{max } & x_1 + x_2 \nonumber \\
    \text{s.a. } & x_1 \in \left\{ 0, 1 \right\}, \tag{EM} \label{eq:EM} \\
    & x_2 \in \left\{ 0, 1 \right\}. \nonumber
  \end{align}
\end{frame}

\section{Enumeração de Soluções}
\begin{frame}{Árvore de enumeração}
  \begin{center}
    \begin{tikzpicture}
      \useasboundingbox (-6,1) rectangle (6,-5);
      \coordinate (root) at (0,0);
      \coordinate (x1=0) at (-2,-2);
      \coordinate (x1=1) at (2,-2);
      \coordinate (00) at (-4.5,-4);
      \coordinate (01) at (-1.5,-4);
      \coordinate (10) at (1.5,-4);
      \coordinate (11) at (4.5,-4);

      \draw<.(2)-> (root) -- (x1=0) node[midway, left] {$x_1 = 0$};
      \draw<.(2)-> (root) -- (x1=1) node[midway, right] {$x_1 = 1$};
      \draw<.(3)-> (x1=0) -- (00) node[midway, left] {$x_2 = 0$};
      \draw<.(3)-> (x1=0) -- (01) node[midway, right] {$x_2 = 1$};
      \draw<.(3)-> (x1=1) -- (10) node[midway, left] {$x_2 = 0$};
      \draw<.(3)-> (x1=1) -- (11) node[midway, right] {$x_2 = 1$};

      \node<.(1)->[fill=white] at (root) {\eqref{eq:EM}};
      \node<.(2)->[fill=white] at (x1=0) {\nodebbbtwo{}{}{}};
      \node<.(2)->[fill=white] at (x1=1) {\nodebbbtwo{}{}{}};
      \node<.(3)->[fill=white] at (00) {\nodebbbtwo{0}{0}{0}};
      \node<.(3)->[fill=white] at (01) {\nodebbbtwo{0}{1}{1}};
      \node<.(3)->[fill=white] at (10) {\nodebbbtwo{1}{0}{1}};
      \node<.(3)->[fill=white] at (11) {\nodebbbtwo{1}{1}{2}};
    \end{tikzpicture}
  \end{center}
\end{frame}

\section{Relaxações}
\begin{frame}{Relação Linear (1)}
  \begin{center}
    \begin{tikzpicture}[scale=2]
      \coordinate (00) at (0,0);
      \coordinate (01) at (0,1);
      \coordinate (10) at (1,0);
      \coordinate (11) at (1,1);

      \draw<.(2)->[fill=gray!50] (00) -- (01) -- (11) -- (10) -- (00);

      \draw[->] (-.4,0) -- (1.4,0) node[right] {$x_1$};
      \draw[->] (0,-.4) -- (0,1.4) node[right] {$x_2$};
      \foreach \x in {00,01,10,11} {
      \fill (\x) circle (.1);
      }

      \draw<.(3)->[color=red] (-.4,.4) -- (.4,-.4);
      \draw<.(3)->[color=red] (-.4,1.4) -- (1.4,-.4);
      \draw<.(3)->[color=red] (.6,1.4) -- (1.4,.6);
      \draw<.(3)->[->] (.5,.5) -- +(.2,.2);
    \end{tikzpicture}
  \end{center}
  \onslide<4>{A solução é $x_1 = 1$ e $x_2 = 1$.}
\end{frame}

\begin{frame}{Relação Linear (2)}
  \begin{center}
    \begin{tikzpicture}[scale=2]
      \coordinate (00) at (0,0);
      \coordinate (01) at (0,1);
      \coordinate (10) at (1,0);
      \coordinate (11) at (1,1);
      \coordinate (opt) at (.8,.6);

      \draw<.(2)->[fill=gray!50] (00) -- (01) -- (opt) -- (10) -- (00);

      \draw[->] (-.4,0) -- (1.4,0) node[right] {$x_1$};
      \draw[->] (0,-.4) -- (0,1.4) node[right] {$x_2$};
      \foreach \x in {00,01,10,11} {
      \fill (\x) circle (.1);
      }

      \draw<.(3)->[color=red] (-.4,.4) -- (.4,-.4);
      \draw<.(3)->[color=red] (-.4,1.4) -- (1.4,-.4);
      \draw<.(3)->[color=red] (.6,1.4) -- (1.4,.6);
      \draw<.(3)->[->] (.5,.5) -- +(.2,.2);
    \end{tikzpicture}
  \end{center}
  \onslide<4>{A solução é $x_1 = 0,8$ e $x_2 = 0,6$.}
\end{frame}

\section{Branch \& Bounds}
\begin{frame}{Motivação}
  \only<1>{
  \begin{center}
    \begin{tikzpicture}[scale=1.5]
      \coordinate (00) at (0,0);
      \coordinate (01) at (0,1);
      \coordinate (10) at (1,0);
      \coordinate (11) at (1,1);
      \coordinate (20) at (2,0);
      \coordinate (21) at (2,1);
      \coordinate (22) at (2,2);
      \coordinate (02) at (0,2);
      \coordinate (12) at (1,2);
      \coordinate (opt) at (1.8,1.6);

      \draw[fill=gray!50] (00) -- (02) -- (12) -- (opt) -- (21) -- (20) -- (00);

      \draw[->] (-.4,0) -- (2.4,0) node[right] {$x_1$};
      \draw[->] (0,-.4) -- (0,2.4) node[right] {$x_2$};
      \foreach \x in {00,01,10,11,20,21,22,02,12} {
      \fill (\x) circle (.1);
      }
    \end{tikzpicture}
  \end{center}
  }
  \onslide<2->{
  \begin{center}
    \begin{tikzpicture}[scale=1.5]
      \coordinate (00) at (0,0);
      \coordinate (01) at (0,1);
      \coordinate (10) at (1,0);
      \coordinate (11) at (1,1);
      \coordinate (20) at (2,0);
      \coordinate (21) at (2,1);
      \coordinate (22) at (2,2);
      \coordinate (02) at (0,2);
      \coordinate (12) at (1,2);
      \coordinate (opt) at (1.8,1.6);

      \draw[fill=gray!50] (00) -- (02) -- (12) -- (10) -- (00);

      \draw[->] (-.4,0) -- (2.4,0) node[right] {$x_1$};
      \draw[->] (0,-.4) -- (0,2.4) node[right] {$x_2$};
      \foreach \x in {00,01,10,11,20,21,22,02,12} {
      \fill (\x) circle (.1);
      }
    \end{tikzpicture}
    \begin{tikzpicture}[scale=1.5]
      \coordinate (00) at (0,0);
      \coordinate (01) at (0,1);
      \coordinate (10) at (1,0);
      \coordinate (11) at (1,1);
      \coordinate (20) at (2,0);
      \coordinate (21) at (2,1);
      \coordinate (22) at (2,2);
      \coordinate (02) at (0,2);
      \coordinate (12) at (1,2);
      \coordinate (opt) at (.8,.6);

      \draw[fill=gray!50] (20) -- (21);

      \draw[->] (-.4,0) -- (2.4,0) node[right] {$x_1$};
      \draw[->] (0,-.4) -- (0,2.4) node[right] {$x_2$};
      \foreach \x in {00,01,10,11,20,21,22,02,12} {
      \fill (\x) circle (.1);
      }
    \end{tikzpicture}
  \end{center}
  }
  \onslide<3>{Evitar o crescimento da árvore de decisão utilizando a solução de
  relaxações do problema.}
\end{frame}

\begin{frame}{Limitantes}
  \begin{block}{Limitante inferior}
    Toda solução factível é um limitante inferior, $\underline{z}$.
  \end{block}

  \begin{block}<.(2)->{Limitante superior}
    Toda solução de uma relaxação é um limitante superior, $\overline{z}$.
  \end{block}
\end{frame}

\begin{frame}{Poda por otimalidade}
  \begin{center}
    \begin{tikzpicture}
      \useasboundingbox (-4,1) rectangle (4,-5);
      \coordinate (s) at (0,0);
      \coordinate (s1) at (-2,-2);
      \coordinate (s2) at (2,-2);
      \coordinate (s21) at (1,-4);
      \coordinate (s22) at (3,-4);

      \draw<.(2)-> (s) -- (s1);
      \draw<.(2)-> (s) -- (s2);
      \draw<.(3)-> (s2) -- (s21);
      \draw<.(3)-> (s2) -- (s22);

      \node<.(1)->[fill=white] at (s) {\nodebounds{s}{27}{13}};
      \node<.(2)->[fill=white, text=red] at (s1) {\nodebounds{s_1}{20}{20}};
      \node<.(2)->[fill=white] at (s2) {\nodebounds{s_2}{25}{15}};
    \end{tikzpicture}
  \end{center}
\end{frame}

\begin{frame}{Poda por limitantes}
  \begin{center}
    \begin{tikzpicture}
      \useasboundingbox (-4,1) rectangle (4,-4);
      \coordinate (s) at (0,0);
      \coordinate (s1) at (-2,-2);
      \coordinate (s2) at (2,-2);
      \coordinate (s21) at (1,-4);
      \coordinate (s22) at (3,-4);

      \draw<.(2)-> (s) -- (s1);
      \draw<.(2)-> (s) -- (s2);
      \draw<.(3)-> (s2) -- (s21);
      \draw<.(3)-> (s2) -- (s22);

      \node<.(1)->[fill=white] at (s) {\nodebounds{s}{27}{13}};
      \node<.(2)->[fill=white] at (s1) {\nodebounds{s_1}{20}{18}};
      \node<.(2)->[fill=white] at (s2) {\nodebounds{s_2}{26}{21}};
    \end{tikzpicture}
  \end{center}
\end{frame}

\begin{frame}{Poda por infactibilidade}
  \begin{center}
    \begin{tikzpicture}
      \useasboundingbox (-4,1) rectangle (4,-4);
      \coordinate (s) at (0,0);
      \coordinate (s1) at (-2,-2);
      \coordinate (s2) at (2,-2);
      \coordinate (s21) at (1,-4);
      \coordinate (s22) at (3,-4);

      \draw<.(2)-> (s) -- (s1);
      \draw<.(2)-> (s) -- (s2);
      \draw<.(3)-> (s2) -- (s21);
      \draw<.(3)-> (s2) -- (s22);

      \node<.(1)->[fill=white] at (s) {\nodebounds{s}{27}{13}};
      \node<.(2)->[fill=white] at (s1) {\nodebounds{s_1}{\emptyset}{\emptyset}};
      \node<.(2)->[fill=white] at (s2) {\nodebounds{s_2}{26}{21}};
    \end{tikzpicture}
  \end{center}
\end{frame}

\begin{frame}{Assuntos não cobertos}
  \begin{itemize}
    \item Relaxações combinatoriais e Lagrangiana,
    \item Seleção da variável para ramificação,
    \item Seleção dos vértices da árvore B\&B a ser resolvido,
    \item Uso de heurísticas para limitante superior, \ldots
  \end{itemize}
\end{frame}

\begin{frame}{Possíveis áreas de pesquisas}
  \begin{itemize}
    \item Relaxações,
    \item Seleção da variável para ramificação,
    \item Seleção dos vértices da árvore B\&B a ser resolvido,
    \item Uso de heurísticas para limitante superior,
    \item Uso de Métodos de Pontos Interiores,
    \item Uso de Computação Paralela,
    \item Programação Não-Linear Inteira, \ldots
  \end{itemize}
\end{frame}<++>
\section*{Obrigado}
\begin{frame}
  \begin{center}
    Obrigado!
  \end{center}
  \begin{center}
    \url{r.gaia.cs@gmail.com}
  \end{center}
\end{frame}

\section*{Bibliografia}
\begin{frame}
  \nocite{wolsey1998integer}
  \bibliography{bbb}
  \bibliographystyle{alpha}
\end{frame}
\end{document}
