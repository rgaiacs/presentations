% Copyright (C) 2012 Raniere Silva
%
% This work is licensed under the Creative Commons
% Attribution-ShareAlike 3.0 Unported License. To view a copy of this
% license, visit <http://creativecommons.org/licenses/by-sa/3.0/>.
%
% This work is distributed in the hope that it will be useful, but
% WITHOUT ANY WARRANTY; without even the implied warranty of
% MERCHANTABILITY or FITNESS FOR A PARTICULAR PURPOSE.

\documentclass[11pt]{beamer}

\usetheme{CambridgeUS}
\let\Tiny=\tiny % Redefine at least \Tiny for avoid warning

\usepackage[utf8]{inputenc}
\usepackage[T1]{fontenc}

\usepackage[brazil]{babel}
\usepackage{indentfirst}
\uselanguage{brazil}
\languagepath{brazil}
\deftranslation[to=brazil]{Example}{Exemplo}

\usepackage{url}
\usepackage{hyperref}
\usepackage{breakurl}

\usepackage{xmpincl}

\usepackage{amsmath}
\usepackage{amsfonts}
\usepackage{amssymb}
\usepackage{amsthm}
\usepackage{breqn}

\usepackage{multicol}
\usepackage{multirow}
\usepackage{array}

\usepackage{graphicx}
\usepackage{subfigure}
\usepackage{wrapfig}
\usepackage{tikz}

\usepackage{algorithmic}
\algsetup{linenosize=\small}

\usepackage{algorithm}
\floatname{algorithm}{Algoritmo}

\usepackage{textcomp}
\usepackage{listings}
\lstset{
% language=Octave,
basicstyle=\ttfamily\scriptsize,
columns=flexible,
% numbers=left,
% numberstyle=\footnotesize,
% stepnumber=5,
% numbersep=5pt,
% backgroundcolor=\color{white},
% showspaces=false,
% showstringspaces=false,
% showtabs=false,
% frame=single,
tabsize=4,
captionpos=t,
breaklines=true,
breakatwhitespace=false,
% caption={\texttt{\lstname}},
% escapeinside={\%*}{*)},
% morekeywords={#},
upquote=true,
}

% Index
\usepackage{makeidx}
\makeindex


\newcommand{\flang}[1]{\textit{#1}}

\hypersetup{
pdftitle={Branch \& Bounds - Básico},
pdfauthor={Raniere Silva},
pdfsubject={Programação Combinatória, Programação Inteira, Branch \& Bounds},
pdfcreator={Raniere Silva},
pdfkeywords={Programação Combinatória, Programação Inteira, Branch \& Bounds},
}
\includexmp{bbb}

\newtheorem{defi}{Definição}
\newtheorem{prop}{Proposição}

\renewcommand{\algorithmicrequire}{\textbf{Entrada:}}
\renewcommand{\algorithmicensure}{\textbf{Saída:}}
\renewcommand{\algorithmicend}{\textbf{fim}}
\renewcommand{\algorithmicif}{\textbf{se}}
\renewcommand{\algorithmicthen}{\textbf{ent\~{a}o}}
\renewcommand{\algorithmicelse}{\textbf{caso contr\'{a}rio}}
\renewcommand{\algorithmicendif}{\algorithmicend}
\renewcommand{\algorithmicfor}{\textbf{para}}
\renewcommand{\algorithmicforall}{\textbf{para todo}}
\renewcommand{\algorithmicdo}{\textbf{fa\c{c}a}}
\renewcommand{\algorithmicendfor}{\algorithmicend}
\renewcommand{\algorithmicwhile}{\textbf{enquanto}}
\renewcommand{\algorithmicendwhile}{\algorithmicend}
\renewcommand{\algorithmicrepeat}{\textbf{repita}}
\renewcommand{\algorithmicuntil}{\textbf{at\'{e}}}
\renewcommand{\algorithmicreturn}{\textbf{retorne}}
\renewcommand{\algorithmiccomment}[1]{\hspace{2em}/* #1 */}

\renewcommand{\lstlistingname}{C\'{o}digo}

\begin{document}
\title[Branch \& Bounds - Básico]{Branch \& Bounds - Básico}
\author[Raniere Silva]{Raniere Gaia Costa da
Silva\footnote{r.gaia.cs@gmail.com}}
\date{22/03/2013}

\begin{frame}
  \maketitle
\end{frame}

\begin{frame}
  \begin{block}{}
    Os arquivos desta apresentação encontram-se disponíveis \\
    em \url{https://github.com/r-gaia-cs/presentations}. \\
    \vspace{-33pt}
    \begin{flushright}
      \includegraphics[height=2cm]{../figures/forkme_right_red.png}
    \end{flushright}
  \end{block}

  \begin{block}{Licença}
    Salvo indicado o contrário, esta apresentação está licenciada sob a
    Licença Creative Commons Atribuição-CompartilhaIgual 3.0 Não Adaptada .
    Para ver uma cópia desta licença, visite
    \url{http://creativecommons.org/licenses/by/3.0/deep.pt_BR}.
    \begin{center}
      \includegraphics{../figures/cc-by-sa.png}
    \end{center}
  \end{block}
\end{frame}

\begin{frame}
    \tableofcontents
\end{frame}

\section{Motivação}
\begin{frame}
  \frametitle{Problemas Reais}
  \begin{itemize}
    \item \flang{Scheduling}
    \item \flang{Planning}
  \end{itemize}
\end{frame}

\begin{frame}
  \frametitle{Exemplos}
  \begin{itemize}
    \item Problema de designação Trabalho-Homem/Máquina
    \item Problema de Cobertura
    \item Problema da Mochila
    \item Problema do Caixeiro Viajante
  \end{itemize}
\end{frame}

\section{Classes de Problemas}
\begin{frame}{PL, PNL e PI}
  \begin{center}
    \begin{tikzpicture}[fill opacity=0.5]
      \filldraw[fill=blue] (120:1.5) circle (2) node[opaque] {PL};
      \pause
      \filldraw[fill=green] (240:1.5) circle (2) node[opaque] {PNL};
      \pause
      \filldraw[fill=red] (0:1.5) circle (2) node[opaque] {PI};
    \end{tikzpicture}
  \end{center}
\end{frame}

\section{Modelagem}
\begin{frame}
  Modelagem Básica
\end{frame}

\section{Enumeração de Soluções}
\begin{frame}
  Enumeração
\end{frame}

\section{Branch \& Bounds}
\begin{frame}
  Motivação
\end{frame}

\section*{Obrigado}
\begin{frame}
  \begin{center}
    Obrigado!
  \end{center}
  \begin{center}
    \url{r.gaia.cs@gmail.com}
  \end{center}
\end{frame}

\section*{Bibliografia}
\begin{frame}
  \nocite{wolsey1998integer}
  \bibliography{bbb}
  \bibliographystyle{alpha}
\end{frame}
\end{document}
