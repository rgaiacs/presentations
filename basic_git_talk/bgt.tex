% Copyright (C) 2012 Raniere Silva
% 
% This work is licensed under the Creative Commons
% Attribution-ShareAlike 3.0 Unported License. To view a copy of this
% license, visit <http://creativecommons.org/licenses/by-sa/3.0/>.
% 
% This work is distributed in the hope that it will be useful, but
% WITHOUT ANY WARRANTY; without even the implied warranty of
% MERCHANTABILITY or FITNESS FOR A PARTICULAR PURPOSE.

\documentclass[11pt]{beamer}
% Utilizar apenas para a classe beamer
\usetheme{CambridgeUS}
\let\Tiny=\tiny % Redefine at least \Tiny for avoid warning

% Tipo de arquivo.
\usepackage[utf8]{inputenc}
% \usepackage[latin1]{inputenc}
\usepackage[T1]{fontenc}

% Configura\c{c}\~{o}es regionais
% \usepackage[top=3cm,left=2cm,right=2cm,bottom=3cm]{geometry}
\usepackage[brazil]{babel}
\usepackage{indentfirst}
\uselanguage{brazil}
\languagepath{brazil}
\deftranslation[to=brazil]{Example}{Exemplo}

% Textos
\newcommand{\flang}[1]{\textit{#1}}

% Links
\usepackage{url}
\usepackage{hyperref}
\hypersetup{
%colorlinks = true,
}
\usepackage{breakurl}

% Pacotes matem\'{a}ticos
\usepackage{amsmath}
\usepackage{amsfonts}
\usepackage{amssymb}
\usepackage{amsthm}
\allowdisplaybreaks[4]
\newtheorem{defi}{Definição}
\newtheorem{prop}{Proposição}

% Pacotes para tabelas
\usepackage{multicol}
\usepackage{multirow}
\usepackage{array}

% Pacotes gr\'{a}ficos
\usepackage{graphicx}
\usepackage{subfigure}
\usepackage{wrapfig}
\usepackage{tikz}
\usetikzlibrary{positioning}
\usetikzlibrary{fit}

% Pacotes para algoritmos
\usepackage{algorithmic}
\algsetup{linenosize=\small}
\renewcommand{\algorithmicrequire}{\textbf{Entrada:}}
\renewcommand{\algorithmicensure}{\textbf{Saída:}}
\renewcommand{\algorithmicend}{\textbf{fim}}
\renewcommand{\algorithmicif}{\textbf{se}}
\renewcommand{\algorithmicthen}{\textbf{ent\~{a}o}}
\renewcommand{\algorithmicelse}{\textbf{caso contr\'{a}rio}}
\renewcommand{\algorithmicendif}{\algorithmicend}
\renewcommand{\algorithmicfor}{\textbf{para}}
\renewcommand{\algorithmicforall}{\textbf{para todo}}
\renewcommand{\algorithmicdo}{\textbf{fa\c{c}a}}
\renewcommand{\algorithmicendfor}{\algorithmicend}
\renewcommand{\algorithmicwhile}{\textbf{enquanto}}
\renewcommand{\algorithmicendwhile}{\algorithmicend}
\renewcommand{\algorithmicrepeat}{\textbf{repita}}
\renewcommand{\algorithmicuntil}{\textbf{at\'{e}}}
\renewcommand{\algorithmicreturn}{\textbf{retorne}}
\renewcommand{\algorithmiccomment}[1]{\hspace{2em}/* #1 */}

\usepackage{algorithm}
\floatname{algorithm}{Algoritmo}

% Pacotes para c\'{o}digos
\usepackage{textcomp}
\usepackage{listings}
\renewcommand{\lstlistingname}{C\'{o}digo}
\lstset{
% language=Octave,
basicstyle=\ttfamily\scriptsize,
columns=flexible,
% numbers=left,
% numberstyle=\footnotesize,
% stepnumber=5,
% numbersep=5pt,
% backgroundcolor=\color{white},
% showspaces=false,
% showstringspaces=false,
% showtabs=false,
% frame=single,
tabsize=4,
captionpos=t,
breaklines=true,
breakatwhitespace=false,
% caption={\texttt{\lstname}},
% escapeinside={\%*}{*)},
% morekeywords={#},
upquote=true,
}
\newcommand{\lcode}[1]{\lstinline!#1!}
% Configura\{c}c\~{o}es para o python
\lstdefinestyle{python}{
language=python,
% escapeinside={\%}{\^{M},
}
\lstnewenvironment{cpython}{\lstset{style=python,}}{}
\newcommand{\fpython}[1]{\lstinputlisting[style=python,]{#1}}

% Index
\usepackage{makeidx}
\makeindex

\begin{document}
\title{Introdução ao Git}
\author[Raniere Silva]{Raniere Gaia Costa da
Silva\footnote{r.gaia.cs@gmail.com}}

\begin{frame}
    \maketitle
\end{frame}

\begin{frame}
    Esta apresentação é baseada na apresentação ``Intro to Git'' de Scott
    Chacon que encontra-se disponível em
    \url{https://github.com/schacon/git-presentations}.

    \begin{block}{}
        Os arquivos desta apresentação encontram-se disponíveis \\
        em \url{https://github.com/r-gaia-cs/presentations}. \\
        \vspace{-33pt}
        \begin{flushright}
            \includegraphics[height=2cm]{../figures/forkme_right_red.png}
        \end{flushright}
    \end{block}

    \begin{block}{Licença}
        Salvo indicado o contrário, esta apresentação está licenciada sob a licença
        Creative Commons Atribuição 3.0 Não Adaptada. Para ver uma cópia desta
        licença, visite http://creativecommons.org/licenses/by/3.0/.
        \begin{center}
            \includegraphics{../figures/cc-by-sa.png}
        \end{center}
    \end{block}
\end{frame}

\begin{frame}
    \tableofcontents
\end{frame}

\section{O que é o Git?}
\begin{frame}
    Git é um sistema de controle de versão distribuido livre
    desenvolvido\footnote{A primeira versão foi escrita por Linus Torvalds, o
    criador do Linux.} para
    ser rápido e eficiente.
\end{frame}

\begin{frame}{Quase tudo é local}
    O Git pode ser utilizado sem conexão de internet ou rede local:
    \begin{itemize}
        \item realizar um \textit{diff}, \pause
        \item visualizar o histórico de um arquivo, \pause
        \item criar uma nova versão (\textit{commit}), \pause
        \item retornar para uma versão anterior, \pause
        \item mudar de ramo, \pause
        \item juntar ramos (\textit{merge}). \pause
    \end{itemize}

    Por este motivo, ele é rápido.
\end{frame}

% O Frame abaixo foi removido para evitar problemas.
%
% \begin{frame}[fragile]{Git é rápido}
%     \begin{lstlisting}
% git clone https://github.com/joyent/node.git -q
% du -sh node
% 160M    node
% cd node; time -p git checkout -b newbranch
% Switched to a new branch 'newbranch'
% real 0.27
% user 0.22
% sys 0.04
% time -p cp -Rf node node2
% real 0.37
% user 0.00
% sys 0.36
%     \end{lstlisting}
% \end{frame}

\section{Work flow}
\begin{frame}[fragile]{Configuração}
    \begin{block}{Arquivos de configuração}
        \begin{enumerate}
            \item  \lstinline+/etc/gitconfig+
            \item  \lstinline+~/.gitconfig+
            \item  \lstinline+.git/config+
        \end{enumerate}
    \end{block}
    \pause
    \begin{lstlisting}
$ git config --global user.name "Raniere Silva"
$ git config --global user.email "r.gaia.cs@gmail.com"
    \end{lstlisting}
    \pause
    \begin{block}{Arquivos a serem ignorados}
        \begin{enumerate}
            \item \lstinline+.gitignore+
        \end{enumerate}
    \end{block}
\end{frame}

\begin{frame}[fragile]{Criando um repositório}
    \begin{lstlisting}
mkdir meu_projeto
cd meu_projeto; git init
Initialized empty Git repository in /home/raniere/documents/presentations/basic_git_talk/src/meu_projeto/.git/
    \end{lstlisting}
\end{frame}

\begin{frame}[fragile]
    \begin{lstlisting}
cd meu_projeto; tree -a
.
+-- .git
    +-- branches
    +-- config
    +-- description
    +-- HEAD
    +-- hooks
    |   +-- applypatch-msg.sample
    |   +-- commit-msg.sample
    |   +-- post-update.sample
    |   +-- pre-applypatch.sample
    |   +-- pre-commit.sample
    |   +-- prepare-commit-msg.sample
    |   +-- pre-rebase.sample
    |   +-- update.sample
    +-- info
    |   +-- exclude
    +-- objects
    |   +-- info
    |   +-- pack
    +-- refs
        +-- heads
        +-- tags

10 directories, 12 files
    \end{lstlisting}
\end{frame}

\begin{frame}[fragile]{Clonando um repositório}
    \begin{lstlisting}
git clone https://github.com/abelsiqueira/tutoriais.git
Cloning into 'tutoriais'...
remote: Counting objects: 207, done.
remote: Compressing objects: 100% (135/135), done.
Receiving objects:  45% (94/207), 236.00 KiB | 203 KiB/s, done.
Receiving objects:  46% (96/207), 348.00 KiB | 202 KiB/s, done.
...
Resolving deltas: 100% (82/82), done.
    \end{lstlisting}
\end{frame}

\begin{frame}{Workflow básico}
    \begin{center}
        \begin{tikzpicture}
            \useasboundingbox (-1,-6.5) rectangle (5,.5);
            \node<.(1)->[draw] at (0,0) (E) {Editar arquivos};
            \node<.(1)->[draw] at (0,-2) (S) {Armazenar as mudanças};
            \draw<.(1)->[->] (E.south) -- (S.north);
            \node<.(2)->[draw] at (0,-4) (R) {Revisar as mudanças};
            \draw<.(2)->[->] (S.south) -- (R.north);
            \node<.(3)->[draw] at (0,-6) (C) {Salvar as mudanças};
            \draw<.(3)->[->] (R.south) -- (C.north);
            \draw<.(4)->[->] (C) -- ++(4,0) -- +(0,6) -- (E);
            \draw<.(4)->[->] (R) -- ++(3.5,0) -- +(0,4) -- (E);
            \draw<.(4)->[->] (S) -- ++(3,0) -- +(0,2) -- (E);
        \end{tikzpicture}
    \end{center}
\end{frame}

\begin{frame}{Estrutura de arquivos (ilustrativa)}
    \begin{center}
        \begin{tikzpicture}
            \useasboundingbox (-1,-6) rectangle (5,1);
            \node<.(2)->[fill=green!50, minimum width=5cm, fit={(foo) (bar)},
            below=4.5cm of root.west, anchor=west] {};
            \node<.(3)->[fill=red!50, minimum width=5cm, fit={(git/index)},
            below=2cm of root.west, anchor=west] {};
            \node<.(4)->[fill=blue!50, minimum width=5cm, fit={(git/repository)},
            below=3cm of root.west, anchor=west] {};
            \node<.(1)->[anchor=west] at (0,0) (root) {./};
            \node<.(1)->[anchor=west] at (1,-1) (git) {.git};
            \node<.(1)->[anchor=west] at (2,-2) (git/index) {index};
            \node<.(1)->[anchor=west] at (2,-3) (git/repository) {repository};
            \node<.(1)->[anchor=west] at (1,-4) (foo) {foo};
            \node<.(1)->[anchor=west] at (1,-5) (bar) {bar};
            \draw<.(1)-> (root) |- (git);
            \draw<.(1)-> (git) |- (git/index);
            \draw<.(1)-> (git) |- (git/repository);
            \draw<.(1)-> (root) |- (foo);
            \draw<.(1)-> (root) |- (bar);
        \end{tikzpicture}
    \end{center}
\end{frame}

\begin{frame}{Workflow básico (comandos)}
    \begin{center}
        \begin{tikzpicture}
            \useasboundingbox (-4,-6) rectangle (4,1);
            \node<.(1)->[fill=green!50, minimum width=5cm] (wd) at (0,0) {Working directory};
            \node<.(1)->[fill=red!50, minimum width=5cm] (id) at (0,-2) {Index};
            \node<.(1)->[fill=blue!50, minimum width=5cm] (rp) at (0,-4) {Repository};
            \draw<.(2)->[->] (wd.south west) to[out=180,in=180] node[midway,fill=white] {add}
            (id.north west);
            \draw<.(3)->[->] (id.south west) to[out=180,in=180] node[midway,fill=white]
            {commit} (rp.north west);
            \draw<.(4)->[->] (rp.north east) to[out=0,in=0] node[midway,fill=white]
            {edit} (wd.south east);
            \draw<.(4)->[->] (id.north east) to[out=0,in=0] node[midway,fill=white]
            {edit} (wd.south east);
        \end{tikzpicture}
    \end{center}
\end{frame}

\begin{frame}[fragile]{Status}
    \begin{lstlisting}
$ git status    
    \end{lstlisting}
\end{frame}

\begin{frame}[fragile]{Add}
    \begin{lstlisting}
$ git add path/to/file
$ git add path/
$ git add .
    \end{lstlisting}
    \pause
    \begin{lstlisting}
$ git add -A|--all
    \end{lstlisting}
    \pause
    \begin{lstlisting}
$ git add -u|--update
    \end{lstlisting}
\end{frame}

\begin{frame}[fragile]{Commit}
    \begin{lstlisting}
$git commit
    \end{lstlisting}
    \begin{lstlisting}
$ git commit -m "Some message"
    \end{lstlisting}
    \pause
    \begin{lstlisting}
$ git commit -m "Some message" -a
    \end{lstlisting}
\end{frame}

\begin{frame}{Ramos}
    \begin{center}
        \begin{tikzpicture}
            \useasboundingbox (-1,-1) rectangle (6,5);
            \node<.(1)->[draw, circle] (C0) at (0,0) {C0};
            \only<1>{
                \draw[->] (1,2) node[fill=gray] {master} -| (C0);
                \draw[->] (1,3) node[fill=gray] {head} -| (C0);
            }
            \node<.(2)->[draw, circle] (C1) at (1,0) {C1};
            \draw<.(2)-> (C0) -- (C1);
            \only<2>{
                \draw[->] (2,2) node[fill=gray] {master} -| (C1);
                \draw[->] (2,3) node[fill=gray] {head} -| (C1);
            }
            \only<3>{
                \draw[->] (2,2) node[fill=gray] {master} -| (C1);
                \draw[->] (2,3) node[fill=gray] {head} -| (C1);
                \draw[->] (2,4) node[fill=gray] {sand} -| (C1);
            }
            \node<.(4)->[draw, circle] (C2) at (2,1) {C2};
            \draw<.(4)-> (C1) -- (C2);
            \only<4>{
                \draw[->] (3,2) node[fill=gray] {master} -| (C1);
                \draw[->] (3,3) node[fill=gray] {head} -| (C2);
                \draw[->] (3,4) node[fill=gray] {sand} -| (C2);
            }
            \node<.(5)->[draw, circle] (C3) at (3,1) {C3};
            \draw<.(5)-> (C2) -- (C3);
            \only<5>{
                \draw[->] (4,2) node[fill=gray] {master} -| (C1);
                \draw[->] (4,3) node[fill=gray] {head} -| (C3);
                \draw[->] (4,4) node[fill=gray] {sand} -| (C3);
            }
            \only<6>{
                \draw[->] (4,2) node[fill=gray] {master} -| (C1);
                \draw[->] (4,3) node[fill=gray] {head} -| (C1);
                \draw[->] (4,4) node[fill=gray] {sand} -| (C3);
            }
            \node<.(7)->[draw, circle] (C4) at (4,0) {C4};
            \draw<.(7)-> (C1) -- (C4);
            \only<7>{
                \draw[->] (5,2) node[fill=gray] {master} -| (C4);
                \draw[->] (5,3) node[fill=gray] {head} -| (C4);
                \draw[->] (5,4) node[fill=gray] {sand} -| (C3);
            }
            \node<.(8)->[draw, circle] (C5) at (5,0) {C5};
            \draw<.(8)-> (C4) -- (C5);
            \draw<.(8)-> (C3) to[out=0] (C5);
            \only<8>{
                \draw[->] (6,2) node[fill=gray] {master} -| (C5);
                \draw[->] (6,3) node[fill=gray] {head} -| (C5);
                \draw[->] (6,4) node[fill=gray] {sand} -| (C5);
            }
        \end{tikzpicture}
    \end{center}
    \only<1>{git init}
    \only<2>{git commit}
    \only<3>{git branch sand}
    \only<4>{git commit}
    \only<5>{git commit}
    \only<6>{git checkout master}
    \only<7>{git commit}
    \only<8>{git merge sand}
\end{frame}

\begin{frame}[fragile]{Log}
    \begin{lstlisting}
$ git log
$ git log -N
    \end{lstlisting}
    \pause
    \begin{lstlisting}
$ git log --oneline
    \end{lstlisting}
    \pause
    \begin{lstlisting}
$ git log --graph
    \end{lstlisting}
\end{frame}

\begin{frame}[fragile]{Diff}
    \begin{lstlisting}
$ git diff
    \end{lstlisting}
    \pause
    \begin{lstlisting}
$ git diff --staged
    \end{lstlisting}
    \pause
    \begin{lstlisting}
$ git diff commit_hash
    \end{lstlisting}
    \pause
    \begin{lstlisting}
$ git diff commit_hash01 commit_hash02
    \end{lstlisting}
\end{frame}

\begin{frame}[fragile]{Branch}
    \begin{lstlisting}
$ git branch new_branch_name
$ git checkout -b new_branch_name
    \end{lstlisting}
    \pause
    \begin{lstlisting}
$ git branch
    \end{lstlisting}
    \pause
    \begin{lstlisting}
$ git branch -a
    \end{lstlisting}
    \pause
    \begin{lstlisting}
$ git branch -d branch_name
    \end{lstlisting}
\end{frame}

\begin{frame}[fragile]{Checkout}
    \begin{lstlisting}
$ git checkout branch_name
    \end{lstlisting}
\end{frame}

\begin{frame}[fragile]{Merge}
    \begin{lstlisting}
$ git merge branch_name
    \end{lstlisting}
\end{frame}

\section{Backup}
\begin{frame}{Prevenção contra acidentes}
    É recomendado manter uma cópia de segurança do repositório para prevenir
    contra:
    \begin{itemize}
        \item roubo/perda do equipamento em que o repositório estava armazenado,
        \item danos irreparáveis no equipamento em que o repositório estava armazenado,
        \item remoção acidental do repositório, \ldots
    \end{itemize}
\end{frame}

\begin{frame}{Serviços de hosting}
    Algumas empresas prestam serviço de hosting para repositórios versionados
    com o Git. Algumas delas são:
    \begin{center}
        \begin{tabular}{p{0.4\textwidth}p{0.2\textwidth}p{0.2\textwidth}}
            Site & Público & Privado \\
            \url{http://github.com/} & Sim & Pago \\
            \url{https://bitbucket.org/}\footnote{Limitado em colaboradores} & Sim & Sim \\
            \url{http://code.google.com/} & Sim & Não \\
            \url{https://gitorious.org/} & Sim & Não
        \end{tabular}
    \end{center}
\end{frame}

\begin{frame}[fragile]{Atualização (1)}
    Para atualizar o backup utiliza-se
    \begin{lstlisting}
$ git push branch_to_upload brach_with_changes
    \end{lstlisting}
\end{frame}

\begin{frame}[fragile]{Atualização (2)}
    Para atualizar o repositório local com modificações ocorridas no backup
    utiliza-se
    \begin{lstlisting}
$ git pull branch_of_backup
    \end{lstlisting}
\end{frame}

\section{Outras possibilidades}
\begin{frame}{Tópicos avançados}
    \begin{enumerate}
        \item Utilizando versões anteriores, \pause
        \item Refazendo último commit, \pause
        \item Alterando a árvore de commits, \pause
        \item Repositórios remotos, \pause
        \item Trabalhos colaborativos, \ldots
    \end{enumerate}
\end{frame}

\begin{frame}{Mais informações}
    \begin{enumerate}
        \item Documentaçãoi (GPL): \url{http://git-scm.com/documentation},
        \item Pro Git (CC-BY-NC-SA): \url{http://git-scm.com/book},
        \item Pragmatic Guide to Git
        \item Pragmatic Version Control Using Git
        \item Version Control with Git: Powerful tools and techniques for
            collaborative software development
    \end{enumerate}
\end{frame}

\section*{Obrigado}
\begin{frame}
    \begin{center}
        Obrigado!
    \end{center}
    \begin{center}
        \url{r.gaia.cs@gmail.com}
    \end{center}
\end{frame}

\end{document}
