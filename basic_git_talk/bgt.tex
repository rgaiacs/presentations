% Copyright (C) 2012 Raniere Silva
% 
% This work is licensed under the Creative Commons
% Attribution-ShareAlike 3.0 Unported License. To view a copy of this
% license, visit <http://creativecommons.org/licenses/by-sa/3.0/>.
% 
% This work is distributed in the hope that it will be useful, but
% WITHOUT ANY WARRANTY; without even the implied warranty of
% MERCHANTABILITY or FITNESS FOR A PARTICULAR PURPOSE.

\documentclass[11pt]{beamer}
% Utilizar apenas para a classe beamer
\usetheme{CambridgeUS}
\let\Tiny=\tiny % Redefine at least \Tiny for avoid warning

% Tipo de arquivo.
\usepackage[utf8]{inputenc}
% \usepackage[latin1]{inputenc}
\usepackage[T1]{fontenc}

% Configura\c{c}\~{o}es regionais
% \usepackage[top=3cm,left=2cm,right=2cm,bottom=3cm]{geometry}
\usepackage[brazil]{babel}
\usepackage{indentfirst}
\uselanguage{brazil}
\languagepath{brazil}
\deftranslation[to=brazil]{Example}{Exemplo}

% Textos
\newcommand{\flang}[1]{\textit{#1}}

% Links
\usepackage{url}
\usepackage{hyperref}
\hypersetup{
%colorlinks = true,
}
\usepackage{breakurl}

% Pacotes matem\'{a}ticos
\usepackage{amsmath}
\usepackage{amsfonts}
\usepackage{amssymb}
\usepackage{amsthm}
\allowdisplaybreaks[4]
\newtheorem{defi}{Definição}
\newtheorem{prop}{Proposição}

% Pacotes para tabelas
\usepackage{multicol}
\usepackage{multirow}
\usepackage{array}

% Pacotes gr\'{a}ficos
\usepackage{graphicx}
\usepackage{subfigure}
\usepackage{wrapfig}
\usepackage{tikz}

% Pacotes para algoritmos
\usepackage{algorithmic}
\algsetup{linenosize=\small}
\renewcommand{\algorithmicrequire}{\textbf{Entrada:}}
\renewcommand{\algorithmicensure}{\textbf{Saída:}}
\renewcommand{\algorithmicend}{\textbf{fim}}
\renewcommand{\algorithmicif}{\textbf{se}}
\renewcommand{\algorithmicthen}{\textbf{ent\~{a}o}}
\renewcommand{\algorithmicelse}{\textbf{caso contr\'{a}rio}}
\renewcommand{\algorithmicendif}{\algorithmicend}
\renewcommand{\algorithmicfor}{\textbf{para}}
\renewcommand{\algorithmicforall}{\textbf{para todo}}
\renewcommand{\algorithmicdo}{\textbf{fa\c{c}a}}
\renewcommand{\algorithmicendfor}{\algorithmicend}
\renewcommand{\algorithmicwhile}{\textbf{enquanto}}
\renewcommand{\algorithmicendwhile}{\algorithmicend}
\renewcommand{\algorithmicrepeat}{\textbf{repita}}
\renewcommand{\algorithmicuntil}{\textbf{at\'{e}}}
\renewcommand{\algorithmicreturn}{\textbf{retorne}}
\renewcommand{\algorithmiccomment}[1]{\hspace{2em}/* #1 */}

\usepackage{algorithm}
\floatname{algorithm}{Algoritmo}

% Pacotes para c\'{o}digos
\usepackage{textcomp}
\usepackage{listings}
\renewcommand{\lstlistingname}{C\'{o}digo}
\lstset{
% language=Octave,
basicstyle=\ttfamily\scriptsize,
columns=flexible,
% numbers=left,
% numberstyle=\footnotesize,
% stepnumber=5,
% numbersep=5pt,
% backgroundcolor=\color{white},
% showspaces=false,
% showstringspaces=false,
% showtabs=false,
% frame=single,
tabsize=4,
captionpos=t,
breaklines=true,
breakatwhitespace=false,
% caption={\texttt{\lstname}},
% escapeinside={\%*}{*)},
% morekeywords={#},
upquote=true,
}
\newcommand{\lcode}[1]{\lstinline!#1!}
% Configura\{c}c\~{o}es para o python
\lstdefinestyle{python}{
language=python,
% escapeinside={\%}{\^{M},
}
\lstnewenvironment{cpython}{\lstset{style=python,}}{}
\newcommand{\fpython}[1]{\lstinputlisting[style=python,]{#1}}

% Index
\usepackage{makeidx}
\makeindex

\begin{document}
\title{Introdução ao Git}
\author[Raniere Silva]{Raniere Gaia Costa da
Silva\footnote{r.gaia.cs@gmail.com}}

\begin{frame}
    \maketitle
\end{frame}

\begin{frame}
    Esta apresentação é baseada na apresentação ``Intro to Git'' de Scott
    Chacon que encontra-se disponível em
    \url{https://github.com/schacon/git-presentations}.

    \begin{block}{}
        Os arquivos desta apresentação encontram-se disponíveis \\
        em \url{https://github.com/r-gaia-cs/presentations}. \\
        \vspace{-33pt}
        \begin{flushright}
            \includegraphics[height=2cm]{../figures/forkme_right_red.png}
        \end{flushright}
    \end{block}

    \begin{block}{Licença}
        Salvo indicado o contrário, esta apresentação está licenciada sob a licença
        Creative Commons Atribuição 3.0 Não Adaptada. Para ver uma cópia desta
        licença, visite http://creativecommons.org/licenses/by/3.0/.
        \begin{center}
            \includegraphics{../figures/cc-by-sa.png}
        \end{center}
    \end{block}
\end{frame}

\begin{frame}
    \tableofcontents
\end{frame}

\section{O que é o Git?}
\begin{frame}
    Git é um sistema de controle de versão distribuido livre
    desenvolvido\footnote{A primeira versão foi escrita por Linus Torvalds, o
    criador do Linux.} para
    ser rápido e eficiente.
\end{frame}

\begin{frame}{Quase tudo é local}
    O Git pode ser utilizado sem conexão de internet ou rede local:
    \begin{itemize}
        \item realizar um \textit{diff}, \pause
        \item visualizar o histórico de um arquivo, \pause
        \item criar uma nova versão (\textit{commit}), \pause
        \item retornar para uma versão anterior, \pause
        \item mudar de ramo (\textit{merge}), \pause
        \item juntar ramos (\textit{merge}). \pause
    \end{itemize}

    Por este motivo, ele é rápido.
\end{frame}

\begin{frame}[fragile]{Git é rápido}
    \begin{lstlisting}
git --version
git version 1.7.9.5
git clone https://github.com/joyent/node.git -q
du -sh node
160M    node
cd node; time -p git checkout -b newbranch
Switched to a new branch 'newbranch'
real 0.27
user 0.22
sys 0.04
time -p cp -Rf node node2
real 0.37
user 0.00
sys 0.36
    \end{lstlisting}
\end{frame}

\section{Work flow}
\begin{frame}[fragile]{Configuração}
    \begin{lstlisting}
$ git config --global user.name "Raniere Silva"
$ git config --global user.email "r.gaia.cs@gmail.com"
    \end{lstlisting}

    \begin{block}{Arquivos}
        \begin{enumerate}
            \item  \lstinline+/etc/gitconfig+
            \item  \lstinline+~/.gitconfig+
            \item  \lstinline+.git/config+
        \end{enumerate}
    \end{block}
\end{frame}

\begin{frame}[fragile]{Criando um repositório}
    \begin{lstlisting}
git --version
git version 1.7.9.5
mkdir meu_projeto
cd meu_projeto; git init
Initialized empty Git repository in /home/raniere/documents/presentations/basic_git_talk/src/meu_projeto/.git/
    \end{lstlisting}
\end{frame}

\begin{frame}[fragile]
    \begin{lstlisting}
cd meu_projeto; tree -a
.
+-- .git
    +-- branches
    +-- config
    +-- description
    +-- HEAD
    +-- hooks
    |   +-- applypatch-msg.sample
    |   +-- commit-msg.sample
    |   +-- post-update.sample
    |   +-- pre-applypatch.sample
    |   +-- pre-commit.sample
    |   +-- prepare-commit-msg.sample
    |   +-- pre-rebase.sample
    |   +-- update.sample
    +-- info
    |   +-- exclude
    +-- objects
    |   +-- info
    |   +-- pack
    +-- refs
        +-- heads
        +-- tags

10 directories, 12 files
    \end{lstlisting}
\end{frame}

\begin{frame}[fragile]{Clonando um repositório}
    \begin{lstlisting}
git --version
git version 1.7.9.5
git clone https://github.com/abelsiqueira/tutoriais.git
Cloning into 'tutoriais'...
remote: Counting objects: 207, done.
remote: Compressing objects: 100% (135/135), done.
Receiving objects:  45% (94/207), 236.00 KiB | 203 KiB/s, done.
Receiving objects:  46% (96/207), 348.00 KiB | 202 KiB/s, done.
...
Resolving deltas: 100% (82/82), done.
    \end{lstlisting}
\end{frame}
\section*{Obrigado}
\begin{frame}
    \begin{center}
        Obrigado!
    \end{center}
    \begin{center}
        \url{r.gaia.cs@gmail.com}
    \end{center}
\end{frame}

\end{document}
