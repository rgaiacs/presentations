% Copyright (C) 2012 Raniere Silva
% 
% This work is licensed under the Creative Commons
% Attribution-ShareAlike 3.0 Unported License. To view a copy of this
% license, visit <http://creativecommons.org/licenses/by-sa/3.0/>.
% 
% This work is distributed in the hope that it will be useful, but
% WITHOUT ANY WARRANTY; without even the implied warranty of
% MERCHANTABILITY or FITNESS FOR A PARTICULAR PURPOSE.

\documentclass{beamer}
% Utilizar apenas para a classe beamer
\usetheme{CambridgeUS}
\let\Tiny=\tiny % Redefine at least \Tiny for avoid warning

% Tipo de arquivo.
\usepackage[utf8]{inputenc}
% \usepackage[latin1]{inputenc}
\usepackage[T1]{fontenc}

% Configura\c{c}\~{o}es regionais
% \usepackage[top=3cm,left=2cm,right=2cm,bottom=3cm]{geometry}
\usepackage[brazil]{babel}
\usepackage{indentfirst}

% Textos
\newcommand{\flang}[1]{\textit{#1}}

% Links
\usepackage{url}
\usepackage{hyperref}
\hypersetup{
%colorlinks = true,
}
\usepackage{breakurl}

% Pacotes matem\'{a}ticos
\usepackage{amsmath}
\usepackage{amsfonts}
\usepackage{amssymb}
\usepackage{amsthm}
\allowdisplaybreaks[4]
\newtheorem{defi}{Definição}
\newtheorem{prop}{Proposição}

% Pacotes para tabelas
\usepackage{multicol}
\usepackage{multirow}
\usepackage{array}

% Pacotes gr\'{a}ficos
\usepackage{graphicx}
\usepackage{subfigure}
\usepackage{tikz}

% Pacotes para algoritmos
\usepackage{algorithmic}
\algsetup{linenosize=\small}
\renewcommand{\algorithmicrequire}{\textbf{Entrada:}}
\renewcommand{\algorithmicensure}{\textbf{Saída:}}
\renewcommand{\algorithmicend}{\textbf{fim}}
\renewcommand{\algorithmicif}{\textbf{se}}
\renewcommand{\algorithmicthen}{\textbf{ent\~{a}o}}
\renewcommand{\algorithmicelse}{\textbf{caso contr\'{a}rio}}
\renewcommand{\algorithmicendif}{\algorithmicend}
\renewcommand{\algorithmicfor}{\textbf{para}}
\renewcommand{\algorithmicforall}{\textbf{para todo}}
\renewcommand{\algorithmicdo}{\textbf{fa\c{c}a}}
\renewcommand{\algorithmicendfor}{\algorithmicend}
\renewcommand{\algorithmicwhile}{\textbf{enquanto}}
\renewcommand{\algorithmicendwhile}{\algorithmicend}
\renewcommand{\algorithmicrepeat}{\textbf{repita}}
\renewcommand{\algorithmicuntil}{\textbf{at\'{e}}}
\renewcommand{\algorithmicreturn}{\textbf{retorne}}
\renewcommand{\algorithmiccomment}[1]{\hspace{2em}/* #1 */}

\usepackage{algorithm}
\floatname{algorithm}{Algoritmo}

% Pacotes para c\'{o}digos
\usepackage{textcomp}
\usepackage{listings}
\renewcommand{\lstlistingname}{C\'{o}digo}
\lstset{
% language=Octave,
basicstyle=\ttfamily\small,
columns=flexible,
numbers=left,
numberstyle=\footnotesize,
stepnumber=5,
numbersep=5pt,
% backgroundcolor=\color{white},
% showspaces=false,
% showstringspaces=false,
% showtabs=false,
% frame=single,
tabsize=4,
captionpos=t,
breaklines=true,
breakatwhitespace=false,
% caption={\texttt{\lstname}},
% escapeinside={\%*}{*)},
% morekeywords={#},
upquote=true,
}
\newcommand{\lcode}[1]{\lstinline!#1!}
% Configura\{c}c\~{o}es para o python
\lstdefinestyle{python}{
language=python,
% escapeinside={\%}{\^{M},
}
\lstnewenvironment{cpython}{\lstset{style=python,}}{}
\newcommand{\fpython}[1]{\lstinputlisting[style=python,]{#1}}

% Index
\usepackage{makeidx}
\makeindex

\begin{document}
\title{Boas Práticas para Computação Científica}
\author[Raniere Silva]{Raniere Gaia Costa da Silva}
\date{30/11/2012}

\begin{frame}
    \maketitle
\end{frame}

\begin{frame}
    Esta apresentação é baseada no artigo ``Best Practices for Scientific
    Computing'' de D. A. Aruliah \textit{et al.} que encontra-se disponível em
    \url{http://arxiv.org/abs/1210.0530v1}.

    \begin{block}{Licença}
        Salvo indicado o contrário, esta apresentação está licenciada sob a licença
        Creative Commons Atribuição 3.0 Não Adaptada. Para ver uma cópia desta
        licença, visite http://creativecommons.org/licenses/by/3.0/.
        \begin{center}
            \includegraphics{../figures/cc-by-sa.png}
        \end{center}
    \end{block}
\end{frame}

\begin{frame}
    \tableofcontents
\end{frame}

\section{Introdução}
\begin{frame}
    \begin{quotation}
        Scientists spend an increasing amount of time building and using
        software. However, most scientists are never taught how to do this
        efficiently. As a result, many are unaware of tools and practices that
        would allow them to write more reliable and maintainable code with less
        effort. We describe a set of best practices for scientific software
        development that have solid foundations in research and experience, and
        that improve scientists’ productivity and the reliability of their
        software. (Retirado de Aruliah\nocite{Aruliah-2012-Best}.)
    \end{quotation}
\end{frame}

\section{Códigos para seres humanos}
\begin{frame}
    Escreva códigos que possa ser facilmente lido e compreendido por outros
    programadores.
\end{frame}

\section{Automatize tarefas repetitivas}
\begin{frame}
    Salve comandos em arquivos para que estes seja reutilizados.
\end{frame}

\section{Use controle de versão}
\begin{frame}
    Git
\end{frame}

\section{Realize testes}
\begin{frame}
    Testes automáticos.
\end{frame}

\begin{frame}
    \bibliography{bpsc}
    \bibliographystyle{alpha}
\end{frame}
\end{document}
