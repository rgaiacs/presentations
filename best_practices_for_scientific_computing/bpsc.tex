% Copyright (C) 2012 Raniere Silva
% 
% This work is licensed under the Creative Commons
% Attribution-ShareAlike 3.0 Unported License. To view a copy of this
% license, visit <http://creativecommons.org/licenses/by-sa/3.0/>.
% 
% This work is distributed in the hope that it will be useful, but
% WITHOUT ANY WARRANTY; without even the implied warranty of
% MERCHANTABILITY or FITNESS FOR A PARTICULAR PURPOSE.

\documentclass[11pt]{beamer}
% Utilizar apenas para a classe beamer
\usetheme{CambridgeUS}
\let\Tiny=\tiny % Redefine at least \Tiny for avoid warning

% Tipo de arquivo.
\usepackage[utf8]{inputenc}
% \usepackage[latin1]{inputenc}
\usepackage[T1]{fontenc}

% Configura\c{c}\~{o}es regionais
% \usepackage[top=3cm,left=2cm,right=2cm,bottom=3cm]{geometry}
\usepackage[brazil]{babel}
\usepackage{indentfirst}
\uselanguage{brazil}
\languagepath{brazil}
\deftranslation[to=brazil]{Example}{Exemplo}

% Textos
\newcommand{\flang}[1]{\textit{#1}}

% Links
\usepackage{url}
\usepackage{hyperref}
\hypersetup{
%colorlinks = true,
}
\usepackage{breakurl}

% Pacotes matem\'{a}ticos
\usepackage{amsmath}
\usepackage{amsfonts}
\usepackage{amssymb}
\usepackage{amsthm}
\allowdisplaybreaks[4]
\newtheorem{defi}{Definição}
\newtheorem{prop}{Proposição}

% Pacotes para tabelas
\usepackage{multicol}
\usepackage{multirow}
\usepackage{array}

% Pacotes gr\'{a}ficos
\usepackage{graphicx}
\usepackage{subfigure}
\usepackage{wrapfig}
\usepackage{tikz}

% Pacotes para algoritmos
\usepackage{algorithmic}
\algsetup{linenosize=\small}
\renewcommand{\algorithmicrequire}{\textbf{Entrada:}}
\renewcommand{\algorithmicensure}{\textbf{Saída:}}
\renewcommand{\algorithmicend}{\textbf{fim}}
\renewcommand{\algorithmicif}{\textbf{se}}
\renewcommand{\algorithmicthen}{\textbf{ent\~{a}o}}
\renewcommand{\algorithmicelse}{\textbf{caso contr\'{a}rio}}
\renewcommand{\algorithmicendif}{\algorithmicend}
\renewcommand{\algorithmicfor}{\textbf{para}}
\renewcommand{\algorithmicforall}{\textbf{para todo}}
\renewcommand{\algorithmicdo}{\textbf{fa\c{c}a}}
\renewcommand{\algorithmicendfor}{\algorithmicend}
\renewcommand{\algorithmicwhile}{\textbf{enquanto}}
\renewcommand{\algorithmicendwhile}{\algorithmicend}
\renewcommand{\algorithmicrepeat}{\textbf{repita}}
\renewcommand{\algorithmicuntil}{\textbf{at\'{e}}}
\renewcommand{\algorithmicreturn}{\textbf{retorne}}
\renewcommand{\algorithmiccomment}[1]{\hspace{2em}/* #1 */}

\usepackage{algorithm}
\floatname{algorithm}{Algoritmo}

% Pacotes para c\'{o}digos
\usepackage{textcomp}
\usepackage{listings}
\renewcommand{\lstlistingname}{C\'{o}digo}
\lstset{
% language=Octave,
basicstyle=\ttfamily\scriptsize,
columns=flexible,
% numbers=left,
% numberstyle=\footnotesize,
% stepnumber=5,
% numbersep=5pt,
% backgroundcolor=\color{white},
% showspaces=false,
% showstringspaces=false,
% showtabs=false,
% frame=single,
tabsize=4,
captionpos=t,
breaklines=true,
breakatwhitespace=false,
% caption={\texttt{\lstname}},
% escapeinside={\%*}{*)},
% morekeywords={#},
upquote=true,
}
\newcommand{\lcode}[1]{\lstinline!#1!}
% Configura\{c}c\~{o}es para o python
\lstdefinestyle{python}{
language=python,
% escapeinside={\%}{\^{M},
}
\lstnewenvironment{cpython}{\lstset{style=python,}}{}
\newcommand{\fpython}[1]{\lstinputlisting[style=python,]{#1}}

% Index
\usepackage{makeidx}
\makeindex

\begin{document}
\title[Boas Práticas para Comp. Científica]{Boas Práticas para Computação Científica}
\author[Raniere Silva]{Raniere Gaia Costa da
Silva\footnote{r.gaia.cs@gmail.com}}
\date{30/11/2012}

\begin{frame}
    \maketitle
\end{frame}

\begin{frame}
    Esta apresentação é baseada no artigo ``Best Practices for Scientific
    Computing'' de D. A. Aruliah \textit{et al.} que encontra-se disponível em
    \url{http://arxiv.org/abs/1210.0530v2}.

    \begin{block}{}
        Os arquivos desta apresentação encontram-se disponíveis \\
        em \url{https://github.com/r-gaia-cs/presentations}. \\
        \vspace{-33pt}
        \begin{flushright}
            \includegraphics[height=2cm]{../figures/forkme_right_red.png}
        \end{flushright}
    \end{block}

    \begin{block}{Licença}
        Salvo indicado o contrário, esta apresentação está licenciada sob a licença
        Creative Commons Atribuição 3.0 Não Adaptada. Para ver uma cópia desta
        licença, visite http://creativecommons.org/licenses/by/3.0/.
        \begin{center}
            \includegraphics{../figures/cc-by-sa.png}
        \end{center}
    \end{block}
\end{frame}

\begin{frame}
    \tableofcontents
\end{frame}

\section{Introdução}
\begin{frame}
    \begin{quotation}
        \textbf<2>{Scientists spend an increasing amount of time building and using
        software}. However, most scientists are never taught how to do this
        efficiently. As a result, many are unaware of tools and practices that
        would allow them to write more reliable and maintainable code with less
        effort. We describe a set of best practices for scientific software
        development that have solid foundations in research and experience, and
        that improve scientists’ productivity and the reliability of their
        software. (Retirado de \cite{Aruliah-2012-Best}.)
    \end{quotation}
\end{frame}

\begin{frame}
    \begin{quotation}
        \textbf<2>{None of these practices will guarantee efficient, error-free, software
        development}, but used in concert they will reduce the number of
        errors in scientific software, make it easier to reuse, and save the
        authors of the software time and effort that can used for focusing on
        the underlying scientific questions. (Retirado de
        \cite{Aruliah-2012-Best}.)
    \end{quotation}
\end{frame}

\section{Códigos para seres humanos}
\begin{frame}{O Código}
    Um código deve:
    \begin{itemize}
        \item executar corretamente e
        \item ser fácil de ler.
    \end{itemize}

    \pause
    Para que o código seja fácil de ler deve-se considerar os seguintes
    aspectos quando este for ser escrito:
    \begin{itemize}
        \item o leitor possui memória e atenção bastante limitada e
        \item uso involuntário de reconhecimento de padrões.
    \end{itemize}
\end{frame}

\begin{frame}[fragile]{Nomes}
    Nomes devem ser consistente, distintos e significativos. Para isso evite
    nomes não-descritivos e muito similares.

    \pause
    \begin{example}[Nomes não-descritivos]
        \verb+a+, \verb+foo+, \verb+bar+, \ldots
    \end{example}

    \pause
    \begin{example}[Nomes similares]
        \verb+resultado+, \verb+resultado1+, \verb+resultados+, \ldots
    \end{example}
\end{frame}

\begin{frame}[fragile]{Estilos}
    Deve-se seguir um único estilo de indentação e preferencialmente não
    misturar \verb+CamelCaseNaming+ com \verb+pothole_case_naming+.

    Exemplos de estilos podem ser encontrados em:
    \begin{description}
        \item[C++]
            \url{http://google-styleguide.googlecode.com/svn/trunk/cppguide.xml}.
        \item[Python]
            \url{http://google-styleguide.googlecode.com/svn/trunk/pyguide.html}.
        \item[R]
            \url{http://google-styleguide.googlecode.com/svn/trunk/google-r-style.html}.
    \end{description}
\end{frame}

\begin{frame}[fragile]{Parâmetros (1)}
    A ordem dos parâmetros é bastante importante para evitar erros por parte
    dos usuários.
    \pause
    \begin{example}
        Considere a subrotina em Fortran abaixo:
        \lstinputlisting[language=Fortran]{src/rect_area.f}
        As seguintes chamadas seriam erradas:
        \begin{lstlisting}
      call rect_area(x_1, y_1, x_2, y_2, a)
      call rect_area(a, x_1, x_2, y_1, y_2)
        \end{lstlisting}
    \end{example}
\end{frame}

\begin{frame}[fragile]{Parâmetros (2)}
    Via de regra o uso de estruturas/objetos é recomendada.
    \pause
    \begin{example}
        Considere a subrotina para o GNU Octave abaixo:
        \lstinputlisting[language=Octave]{src/sum_vect.m}
        A seguinte chamada seria errada:
        \begin{lstlisting}
a = [0 0];
b = [1 1];
x = sum_vect(a, b);
        \end{lstlisting}
    \end{example}
\end{frame}

\section{Automatize tarefas repetitivas}
\begin{frame}{Evite repetir comandos e clicks (1)}
    Deve-se evitar repetir comandos e clicks utilizados com muita frequência.

    Em relação aos comandos, recomenda-se utilizar o GNU Make e/ou alguma
    linguagem de script, e.g.,
    \begin{itemize}
        \item GNU Bash,
        \item Python,
        \item Ruby,
        \item GNU Octave, \ldots
    \end{itemize}

    E em relação aos clicks, recomenda-se atribuí-los a algum novo botão ou
    atalho.
\end{frame}

\begin{frame}[fragile]{Evite repetir comandos e clicks (2)}
    \begin{example}
        Considere o código abaixo em C:
        \lstinputlisting[language=C]{src/hello.c}
        Para a tarefa de compilá-lo e executá-lo podemos utilizar o GNU Make
        por meio do arquivo abaixo:
        \lstinputlisting[language=bash]{src/Makefile}
    \end{example}
\end{frame}

\begin{frame}[fragile]{Evite repetir comandos e clicks (3)}
    \begin{example}[Continuação]
         E por fim,  procedemos com
        \begin{lstlisting}
$ make run
gcc hello.c -o hello
./hello
Hello.
$ make run
./hello
Hello.
        \end{lstlisting}
    \end{example}
\end{frame}

\begin{frame}{Use o histórico}
    \begin{quotation}
        (\ldots) most command-line tools have a ``history'' option that
        lets users display and re-execute recent commands, with minor edits to
        filenames or parameters. This is often cited as \textbf{one reason command-line
        interfaces remain popular}. (Retirado de \cite{Aruliah-2012-Best}.)
    \end{quotation}
\end{frame}

\begin{frame}{Permita a reprodução}
    \begin{quotation}
        In order to maximize reproducibility, everything needed to re-create the
        output should be recorded (\dots). Retirado de \cite{Aruliah-2012-Best}.)
    \end{quotation}
\end{frame}

\section{Use controle de versão}
\begin{frame}{Desafios}
    \begin{quotation}
        Two of the \textbf<2>{biggest challenges} scientists and other programmers face
        when working with code and data are \textbf<2>{keeping track of changes} (and being
        able to revert them if things go wrong), and \textbf<2>{working collaboratively} on
        a program or dataset. (Retirado de \cite{Aruliah-2012-Best}.)
    \end{quotation}
\end{frame}

\begin{frame}{Comparativos}
    \begin{center}
        \begin{tabular}{p{.25\textwidth}p{.2\textwidth}p{.2\textwidth}p{.2\textwidth}}
            & \textbf{email} & \textbf{\textit{shared folder}} &
            \textbf{DVCS} \uncover<2->{\\
            \textbf{Notificação} & Fácil & Difícil & Média} \uncover<3->{\\
            \textbf{Identificação} & Difícil & Fácil & Fácil} \uncover<4->{\\
            \textbf{Desfazer} & Difícil & Difícil & Fácil} \uncover<5->{\\
            \textbf{Procurar} & Difícil & Difícil & Fácil}
        \end{tabular}
    \end{center}
    \uncover<0->{
    \begin{description}
        \item[shared folder] uso de disco compartilhado por rede local ou
            serviço nas nuvens, e.g., Dropbox.
        \item[DCSV] \textit{Distributed Concurrent Versions System}, e.g., git,
            mercurial (hg).
    \end{description}
    }
\end{frame}

\begin{frame}[fragile]{Git}
    \begin{example}
        \begin{lstlisting}[escapeinside=()]
git init -q(\pause)
echo 'Versao 0.1' >> meu_projeto.txt(\pause)
git add meu_projeto.txt(\pause)
git commit -q -m 'Iniciei'(\pause)
echo 'Versao 0.2' >> meu_projeto.txt(\pause)
git add -u(\pause)
git commit -q -m 'Mudei'(\pause)
cat meu_projeto.txt
Versao 0.1
Versao 0.2(\pause)
git checkout HEAD^ -q(\pause)
cat meu_projeto.txt
Versao 0.1(\pause)
git checkout master -q(\pause)
cat meu_projeto.txt
Versao 0.1
Versao 0.2
        \end{lstlisting}
        % \lstinputlisting[language=bash]{src/git.sh.out}
    \end{example}
\end{frame}

\section{Realize testes}
\begin{frame}{Teste}
    Realizar testes para verificar a execução correta do código é essencial.
    
    Fazê-los de maneira eficiêncie é uma arte.

    \pause
    Os dois tipos principais de testes são:
    \begin{itemize}
        \item White-Box testing e
        \item Black-box testing.
    \end{itemize}

    \pause
    Uma de lista \textit{frameworks} para teste encontra-se em
    \url{http://en.wikipedia.org/wiki/List_of_unit_testing_frameworks}.
\end{frame}

\begin{frame}[fragile]{Python Doctest (1)}
    \begin{example}
        \lstinputlisting[language=Python]{src/sum_vect.py}
    \end{example}
\end{frame}

\begin{frame}[fragile]{Python Doctest (2)}
    \begin{example}[Continuação]
        \lstinputlisting[language=bash]{src/sum_vect.py.make.out}
    \end{example}
\end{frame}

\section{Por onde começar?}
\begin{frame}
    Aos poucos ir tornando cada uma das boas práticas um hábito.
    \begin{block}{Atenção}
        Não deve-se seguir a risca as soluções aqui propostas.
    \end{block}
\end{frame}

\section{Futuros trabalhos}
\begin{frame}
    Pretende-se apresentar o Git com maiores detalhes entre Janeiro e Fevereiro
    de 2013.

    O material para a apresentação já começou a ser desenvolvido e a versão
    beta encontra-se em
    \url{https://github.com/downloads/r-gaia-cs/presentations/bgt.pdf}.
\end{frame}

\section*{Obrigado}
\begin{frame}
    \begin{center}
        Obrigado!
    \end{center}
    \begin{center}
        \url{r.gaia.cs@gmail.com}
    \end{center}
\end{frame}

\section*{Bibliografia}
\begin{frame}
    \bibliography{bpsc}
    \bibliographystyle{alpha}
\end{frame}
\end{document}
